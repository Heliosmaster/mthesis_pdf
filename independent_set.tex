\chapter{Maximum independent set formulation of the partial row/column assignment problem} \label{chap:independent_set}

With the framework introduced in Section \ref{sec:hot_restart}, we basically translated the problem of the assignment of nonzeros to $A_r$ and $A_c$ (which is already another formulation of the matrix partitioning problem with the medium grain model) to the problem of an efficient computation of a permutation of the indices $\{0,\dots,m+n-1\}$. In this chapter, we will propose a method for this vector computation problem which relies on concepts of the field of graph theory.

The main idea is somewhat similar to the principle that led us to the development of the Separated Block Diagonal form of order 2 in Section \ref{sec:sbd2}. In that particular form of a partitioned matrix, the blocks $\ddot{A}_{00}$ and $\ddot{A}_{44}$ are interesting, as they contain ``independent'' nonzeros. More specifically, those rows and columns are fully assigned to a processor, and whose nonzeros do not have any neighbor (a nonzero in the same row or column) which has a cut column/row. The analogue of this concept of independence, is now to be defined carefully: we want to find a subset of the indices $\{0,\dots,m+n-1\}$ which does not cause any communication, whenever we fully assign its rows to $A_r$ and its columns to $A_c$. With this definition, our goal is clear: we want to assign as many nonzeros as possible in this way, obtaining a low upper bound on the communication volume, which can be computed during the creation of $A_r$ and $A_c$.

To do so, we can employ a well studied object in graph theory: the \textbf{maximum independent set}. However, this requires a correct translation of our sparse matrix into a graph, described in Section \ref{sec:is_graph}. In Section \ref{sec:is_comp}, we delve a little more into the graph theory required and describe the actual algorithm used to compute a maximum independent set in such a graph. In Section \ref{sec:is_vector}, finally, we give a few different possibilities for computing the priority vector $v$ using the concepts and algorithms just introduced.

\section{Graph construction} \label{sec:is_graph}

We need to construct the graph correctly from our sparse matrix, in order to retrieve the desired information. In our case, we can simply consider the graph whose adjacency matrix is none other than the sparsity pattern of our matrix $A$. This exact same formulation has already been studied, for example, by Hendrickson and Kolda \cite{hendrickson}, who used their \emph{bipartite graph model} to discuss different algorithms for bipartite graph partitioning.

More explicitly, in this graph formulation, rows and columns are vertices, and we have the edge $(i,j)$ if $a_{ij} \neq 0$. It is fairly clear that the resulting graph is bipartite, because an edge connects only a row with a column.

An example of such translation from matrix to graph is shown in Figure \ref{fig:bipartite_graph}, where we start from the matrix given in Figure \ref{fig:partition}. 

\begin{figure}[h]
	\centering
	\begin{tikzpicture}[scale=0.5]
		\foreach \x / \y in {1/1,1/3,2/2,4/5,4/4,7/3,8/5,6/7,9/1} { \fill[myred] ({\y-1},{-\x+1}) rectangle +(1,-1);}
		\foreach \x / \y in {1/2,2/3,3/6,3/9,6/6,5/1,7/8,8/7,9/2} { \fill[myblue] ({\y-1},{-\x+1}) rectangle +(1,-1);}
%		\draw[semithick] (0,-9) grid (9,0);
		\draw[thick] (0,-9) rectangle (9,0);

		\draw[myarrow,thick] (10,-4.5) -- (13,-4.5);

		\foreach \x in {0,...,8} { 
			\node[vertex,label=left:\(r_{\x}\)] (r\x) at (16,{3.5-2*\x}) {}; 
			\node[vertex,label=right:\(c_{\x}\)] (c\x) at (26,{3.5-2*\x}) {};
			\node at (-1,{-\x-0.5}) {\x};
			\node at (\x+0.5,1) {\x};
		}

		\foreach \x / \y in {0/0,0/2,1/1,3/4,3/3,6/2,7/4,5/6,8/0} { \draw[myred] (r\x) -- (c\y);}
		\foreach \x / \y in {0/1,1/2,2/5,2/8,5/5,4/0,6/7,7/6,8/1} { \draw[myblue] (r\x) -- (c\y);} 
	\end{tikzpicture}
	\caption{Graph constructed using the sparsity pattern of the matrix of Figure \ref{fig:partition} as adjacency matrix (rows and columns are vertices, nonzeros are edges). The edge color has been kept the same as the corresponding nonzero, but only to facilitate the understanding. The fact that the matrix is partitioned does not play any role in the resulting graph. In the bipartite graph, with $r_i$ we denote row $i$, whereas with $c_j$ we denote column $j$.} \label{fig:bipartite_graph}
\end{figure}

\section{The maximum independent set and its computation} \label{sec:is_comp}

In this section, we will give an overview of the maximum independent set problem, discuss its complexity and the relation with other famous problems in graph theory, and, lastly, give an efficient algorithm that can be used with a bipartite graph.

\subsection{Maximum independent set}

The concepts of \emph{independent set} and \emph{vertex cover} are closely related \cite{np_book}: let $G=(V,E)$ be an undirected graph.

\begin{definition}[Independent set]
	An \emph{independent set} is a subset $V' \subseteq V$ such that $ \forall u,v \in V'$, $(u,v) \notin E$.
	A \emph{maximum} independent set is an independent set of $G$ with maximum cardinality.
\end{definition}

\begin{definition}[Vertex cover]
	A \emph{vertex cover} is a subset $V' \subseteq V$ such that $\forall (u,v) \in E$ we have $u \in V' \vee v \in V'$, i.e. at least one of the endpoints of any edge is in the cover. A \emph{minimum} vertex cover is a vertex cover of $G$ with minimum cardinality.
\end{definition}

A graphical depiction of two independent sets for an example graph is shown in Figure \ref{fig:is_example}.

\begin{figure}[h]
	\centering
\begin{tikzpicture}[scale=0.2]
	\tikzstyle{vertex} = [fill,shape=circle,node distance=100pt,minimum size=0.3cm,inner sep = 0pt]
		\foreach \x / \y / \z in {17/0/1,12/12/2,-17/0/5,-12/12/4,12/-12/8,0/17/3,0/-17/7,-12/-12/6,0/0/0} { \node[vertex] (n\z) at (\x,\y) {};}
		\foreach \x in {2,4,...,8} { \draw (n\x) -- (n0);}

		\draw (n1) -- (n2) -- (n3) -- (n4) -- (n5) -- (n6) -- (n7) -- (n8) -- (n1);
		\draw (n8) -- (n3) -- (n6);
		\draw (n2) -- (n7) -- (n4);
		\draw (n2) -- (n5) -- (n8);
		\draw (n4) -- (n1) -- (n6);

		\foreach \x in {2,4,6,8} {\node[vertex,myred] at (n\x) {}; }

		\foreach \x / \y / \z in {17/0/1,12/12/2,-17/0/5,-12/12/4,12/-12/8,0/17/3,0/-17/7,-12/-12/6,0/0/0} { \node[vertex] (m\z) at (42+\x,\y) {};}
		\foreach \x in {2,4,...,8} { \draw (m\x) -- (m0);}

		\draw (m1) -- (m2) -- (m3) -- (m4) -- (m5) -- (m6) -- (m7) -- (m8) -- (m1);
		\draw (m8) -- (m3) -- (m6);
		\draw (m2) -- (m7) -- (m4);
		\draw (m2) -- (m5) -- (m8);
		\draw (m4) -- (m1) -- (m6);
		\foreach \x in {0,1,3,5,7} {\node[vertex,myred] at (m\x) {}; }

	\end{tikzpicture}
	\caption{Two different independent sets (red vertices) on an example graph. The two independent sets have different cardinality, and the one on the right is a maximum independent set.} \label{fig:is_example}
\end{figure}

The following lemma explicitly gives us the relation between a vertex cover and an independent set.

\begin{lemma} 
	\label{lemma:is}
	Given a graph $G$, $V'$ is a vertex cover set if and only if $V \setminus V'$ is an independent set.
\end{lemma}
\begin{proof}
	Let $V'$ be a vertex cover, i.e. $\forall (u,v) \in E$, $u \in V'$ or $v \in V'$. This is equivalent to say that $\forall u,v \in V \setminus V'$, $(u,v) \notin E$, which is the definition of independent set.
\end{proof}

As the decision variant of the problem of finding a minimum vertex cover is NP-complete \cite[Theorem 3.3]{np_book}, it follows from this lemma that also finding a maximum independent set in a graph is NP-complete; the main consequence of this result is that we cannot solve this problem directly for a generic graph, as it would be as hard as our original matrix partitioning problem. Luckily, we are dealing with a particular kind of graph, a bipartite graph, which simplifies greatly the computations of a maximum independent set.

Before exploiting the bipartiteness of our graph, we need to make an additional observation: Lemma \ref{lemma:is} states that, in a generic graph, the vertex cover problem and independent set cover are complementary. Therefore, computing a maximum independent set is equivalent to computing a minimum vertex cover. This equivalence is particularly useful in our case, because another mathematical object can be related to the minimum vertex cover as well: the maximum matching.

\begin{definition}[Matching]
	Let $G=(V,E)$ be a graph. A \emph{matching} $M \subseteq E$ is a set of edges such that at most one edge from $M$ is incident to each vertex $v \in V$. We say that a vertex $v \in V$ is matched by $M$ if an edge in $M$ is incident to $v$. A \emph{maximum} matching is a matching of maximum cardinality.
\end{definition}

In particular, because we are in a bipartite graph, we can employ K\H{o}nig's Theorem \cite{konig}:

\begin{theorem}[K\H{o}nig]
In a bipartite graph, the size of a maximum matching is equal to the size of a minimum vertex cover. \label{th:konig}
\end{theorem}

We will algorithmically prove the theorem, showing that from a maximum matching we can obtain a minimum vertex cover and their size is equal. First, however, we need two more definitions that are useful when dealing with (maximum) matchings.

\begin{definition}[Simple path]
	Let $G=(V,E)$ be a graph. A path $P=(v_1,\dots,v_k)$ is said to be \emph{simple} if $v_i \neq v_j$, $\forall i \neq j$, i.e. all the vertices are distinct and there are no self-edges or sub-cycles. \label{def:simple_path} 
\end{definition}

\begin{definition}[Augmenting path]
	Let $M$ be a matching on the graph $G=(V,E)$. The simple path $P$ is said to be \emph{augmenting} if it starts and ends on unmatched (or exposed) vertices, and its edges alternate between $E \setminus M$ and $M$, in this order.\label{def:augmenting_path}
\end{definition}

It is easy to see that, if we have a matching $M$ and an augmenting path $P$, if $P$ contains $k$ edges in $M$, then it has exactly $k+1$ edges in $E\setminus M$, and, if the graph is bipartite, the two endpoints of $P$ belong to the two different sets of vertices. Moreover, $M \oplus P$ is a matching of size $|M|+1$, where $M \oplus P := (M \setminus P) \cup (P \setminus M)$ denotes the \emph{symmetric difference} between $M$ and $P$. 

Now, in Algorithm \ref{alg:directed_graph}, we give a scheme that constructs a bipartite directed graph, starting from a bipartite undirected graph $G=(L \cup R,E)$ and a matching $M$.

\begin{algorithm}[h]
	\begin{algorithmic}
	\Require{Bipartite undirected graph $G=(L \cup R,E)$, matching $M$.}
	\Ensure{Bipartite directed graph $G' = (L \cup R,D)$.}
	\State $D \gets \varnothing$
	\For{$(i,j) \in E$ such that $i \in L$ and $j \in R$}
	\If{$(i,j) \in M$}
	\State $D \gets D \cup (j,i)$
	\Else
	\State $D \gets D \cup (i,j)$
	\EndIf
	\EndFor
	\end{algorithmic}
	\caption{Construction of a bipartite directed graph starting from an undirected bipartite graph and a matching.} \label{alg:directed_graph}
\end{algorithm}

In other words, the edges in $E$ are given a direction: the ones in the matching go from $R$ to $L$, and the others from $L$ to $R$.

In Algorithm \ref{alg:konig}, we give an explicit scheme to compute a minimum vertex cover starting from a maximum matching.

\begin{algorithm}[h]
	\begin{algorithmic}
	\Require{Bipartite graph $G=(L \cup R,E)$, maximum matching $M$.}
	\Ensure{Minimum vertex cover $C$.}
	\State Construct the modified graph $G' = (L \cup R,D)$ as in Algorithm \ref{alg:directed_graph}
	\State $T \gets \varnothing$
	\ForAll{$v \in L$ such that $v$ is not matched}
	\State Add $v$ to $T$
	\ForAll{$u \in L \cup R$ reachable from $v$ using the directed edges in $D$}
	\State Add $u$ to $T$
	\EndFor
	\EndFor
	\State $C \gets (L \setminus T) \cup (R \cap T)$
	\end{algorithmic}
	\caption{Construction of the minimum vertex cover in a bipartite graph, starting from the maximum matching.} \label{alg:konig}
\end{algorithm}

In Figure \ref{fig:konig}, we can visualize an example of Algorithm \ref{alg:konig}, starting from the graph of Figure \ref{fig:bipartite_graph}, with a maximum matching. In the final image, the relationships between maximum independent set, minimum vertex cover and maximum matching can be easily recognized. 

\begin{figure}[h]
	\centering
	\subfigure[Graph at the beginning of the algorithm. The edges in blue belong to the maximum matching $M$. $T = \varnothing$.]{ 
		\begin{tikzpicture}[scale=0.5]
			\foreach \x in {0,...,8} { 
				\node[vertex,label=left:\(r_{\x}\)] (r\x) at (16,{3.5-2*\x}) {}; 
				\node[vertex,label=right:\(c_{\x}\)] (c\x) at (26,{3.5-2*\x}) {};
			}

			\foreach \x / \y in {0/0,3/4,6/2,8/0} { \draw (r\x) -- (c\y);}
			\foreach \x / \y in {0/1,1/2,2/8,5/5,7/6,8/1} { \draw (r\x) -- (c\y);}

			\foreach \x / \y in {4/0,1/1,0/2,3/3,7/4,2/5,5/6,6/7} { \draw[myblue,very thick] (r\x) -- (c\y);}
		\end{tikzpicture}
	} \hspace{2cm}
	\subfigure[We start with the unmatched vertex in $r_8$; we traverse the unmatched edge $(r_8,c_0)$ and the matched edge $(c_0,r_4)$. $T = \{r_8,c_0,r_4\}$. ]{ 
		\begin{tikzpicture}[scale=0.5]
			\foreach \x in {0,...,8} { 
				\node[vertex,label=left:\(r_{\x}\)] (r2\x) at (16,{3.5-2*\x}) {}; 
				\node[vertex,label=right:\(c_{\x}\)] (c2\x) at (26,{3.5-2*\x}) {};
			}

			\foreach \x / \y in {0/0,3/4,6/2} { \draw (r2\x) -- (c2\y);}
			\foreach \x / \y in {0/1,1/2,2/8,5/5,7/6,8/1} { \draw (r2\x) -- (c2\y);}

			\foreach \x / \y in {1/1,0/2,3/3,7/4,2/5,5/6,6/7} { \draw[myblue,very thick] (r2\x) -- (c2\y);}
			\foreach \x / \y in {8/0} { \draw[myred] (r2\x) -- (c2\y);}
			\foreach \x / \y in {4/0} { \draw[myred,very thick] (r2\x) -- (c2\y);}
			\foreach \x in {4,8} { \node[vertex,myred] at (r2\x) {};}
			\foreach \x in {0} { \node[vertex,myred] at (c2\x) {};}

		\end{tikzpicture}
	}\vspace{1cm}
	\subfigure[We traverse the unmatched edge $(r_8,c_1)$, then the matched edge $(c_1,r_1)$, then the unmatched $(r_1,c_2)$ and lastly the matched $(c_2,r_0)$. $T = \{ r_8,c_0,r_4,c_1,r_1,c_2, r_0 \}$.]{
		\begin{tikzpicture}[scale=0.5]
			\foreach \x in {0,...,8} { 
				\node[vertex,label=left:\(r_{\x}\)] (r3\x) at (16,{3.5-2*\x}) {}; 
				\node[vertex,label=right:\(c_{\x}\)] (c3\x) at (26,{3.5-2*\x}) {};
			}

			\foreach \x / \y in {0/0,3/4,6/2} { \draw (r3\x) -- (c3\y);}
			\foreach \x / \y in {0/1,8/0,2/8,5/5,7/6} { \draw (r3\x) -- (c3\y);}

			\foreach \x / \y in {4/0,3/3,7/4,2/5,5/6,6/7} { \draw[myblue,very thick] (r3\x) -- (c3\y);}
			\foreach \x / \y in {8/1,1/2} { \draw[myred] (r3\x) -- (c3\y);}
			\foreach \x / \y in {1/1,0/2} { \draw[myred,very thick] (r3\x) -- (c3\y);}
			\foreach \x in {0,1,4,8} { \node[vertex,myred] at (r3\x) {};}
			\foreach \x in {0,1,2} { \node[vertex,myred] at (c3\x) {};}
		\end{tikzpicture}
	}\hspace{2cm}
	\subfigure[We take as $C = (L \setminus T) \cup (R \cap T)$ (depicted in orange). We can see from the final graph that it is indeed a minimum vertex cover.]{
	\begin{tikzpicture}[scale=0.5]

		\foreach \x in {0,...,8} { 
			\node[vertex,label=left:\(r_{\x}\)] (r\x) at (16,{3.5-2*\x}) {}; 
			\node[vertex,label=right:\(c_{\x}\)] (c\x) at (26,{3.5-2*\x}) {};
		}

		\foreach \x / \y in {0/0,3/4,6/2,8/0} { \draw (r\x) -- (c\y);}
		\foreach \x / \y in {0/1,1/2,2/8,5/5,7/6,8/1} { \draw (r\x) -- (c\y);}

		\foreach \x / \y in {4/0,1/1,0/2,3/3,7/4,2/5,5/6,6/7} { \draw[myblue,very thick] (r\x) -- (c\y);}
		\foreach \x in {2,3,5,6,7} { \node[vertex,orange] at (r\x) {};}
		\foreach \x in {0,1,2} { \node[vertex,orange] at (c\x) {};}
	\end{tikzpicture}
	}
	\caption{Example of the actions performed in Algorithm \ref{alg:konig} for the graph of Figure \ref{fig:bipartite_graph}. The red vertices are in the set $T$. In the last image the orange vertices belong to a minimum vertex cover. Lemma \ref{lemma:is} can quickly be checked, as the black vertices are indeed a maximum independent set.} \label{fig:konig}
\end{figure}

With the following lemma, we will prove the correctness of Algorithm \ref{alg:konig} and K\H{o}nig's Theorem.

\begin{lemma}
	Let $G=(L\cup R,E)$ be a bipartite graph and let $M$ be a maximum matching. The subset of vertices $C$ obtained following Algorithm \ref{alg:konig} is the minimum vertex cover. In addition, we have that $|C| = |M|$.
\end{lemma}
\begin{proof}
First of all, we will prove that $C$ is a vertex cover: assume it is not, which means that there is an edge $e = (i,j) \in E$ with $i,j \notin C$ ($i \in L$, $j \in R$), which implies that $i \in L \cap T$ and $j \in R \setminus T$, because the graph is bipartite. 
	
We now have two possibilities: either $e \notin M$ or $e \in M$.  In the first case, because $i \in T$ and $e$ can be traversed (it goes from $L$ to $R$ in the directed graph), $j$ can be reached and thus $j \in T$, a contradiction. In the second case, we have that $i$ is matched and belongs to $T$: this means that it was reached by traversing the matched edge, which implies that $j \in T$, again a contradiction. Since in both cases we get to a contradiction, $C$ is indeed a vertex cover of $G$.

Now, we will prove that $|C| \leq |M|$, by showing that every vertex in $C$ is matched. It is clear that every vertex in $L \setminus T$ is matched, by definition of $T$. Suppose there is a unmatched vertex $v \in R \cap T$: since $v \in T$, it means that it was reached from an unmatched vertex $u$ in $L$, thus the path from $u$ to $v$ is augmenting, which would imply that the matching $M$ is not of maximum cardinality, a contradiction. Moreover, note that there are no edges in the matching between $L \setminus T$ and $R \cap T$ (otherwise, as shown previously, the endpoint in $L$ would also be in $T$).

Therefore we have that all the vertices in $C$ are matched and the edges of the matching are distinct, which implies that $|C| \leq |M|$. Furthermore, for any matching $M'$ and vertex cover $C'$ it is true that $|M'| \leq |C'|$ as there is at least one endpoint in $C'$ for every edge in $M'$. So we have that $|C| = |M|$, proving K\H{o}nig's Theorem, along the way.

Lastly, these inequality also imply that $C$ is the vertex cover of minimum cardinality: assume it is not, i.e. we have that $C'$ is a vertex cover with $|C'| < |C|$. Now, if we consider the maximum matching $M$, the inequalities give us that $|M| \leq |C'| < |C| \leq |M|$, a contradiction. $C$ is then the minimum vertex cover.

\end{proof}

We have shown that there are close relationships between the maximum matching, minimum vertex cover and maximum independent set. Now, the problem is shifted toward finding an efficient way of computing the maximum matching. In the next section we will describe the Hopcroft-Karp algorithm, which is a simple extension of Algorithm \ref{alg:konig}.

\subsection{The Hopcroft-Karp algorithm for bipartite matching} \label{sec:hopcroft_karp}

The Hopcroft-Karp algorithm \cite{hopcroft_karp}, devised in 1973, is an efficient scheme for finding a maximum independent set on bipartite graphs, with a running time of $\mathcal{O}\left( |E|\sqrt{|V|} \right)$. This is a considerable improvement over the famous Ford-Fulkerson algorithm of 1956, which, for bipartite graphs, has a running time of $\mathcal{O}\left( |V||E| \right)$. We can compare these two algorithms, even though the latter is technically meant for maximum flow problems, because a bipartite graph can be modified in such a way that a maximum flow corresponds to a maximum matching in the original graph.

Both algorithms rely on the concept of augmenting paths, introduced in the previous section in Definition \ref{def:augmenting_path}.

The main idea of the Hopcroft-Karp algorithm is to find these augmenting paths to progressively increase the size of the matching, as outlined in Algorithm \ref{alg:hopcroft-karp}. The fact that in the main loop we augment the matching over several augmenting paths simultaneously gives us the $\sqrt{|V|}$ factor in the running time, instead of a simple $|V|$.

\begin{algorithm}[h]
	\begin{algorithmic}
		\Require{Bipartite graph $G=(L \cup R,E)$}
		\Ensure{Maximum matching $M$}
		\State $M \gets \varnothing$
		\Repeat
		\State $l_M \gets$ length of the shortest augmenting path, using the matching $M$
		\State $P \gets \{ P_1,\dots,P_k \}$, a maximal set of vertex-disjoint shortest augmenting paths of length $l_M$
		\State $M \gets M \oplus (P_1 \cup \dots \cup P_k )$
		\Until{ $P=\varnothing$}
	\end{algorithmic}
	\caption{Basic outline of the Hopcroft-Karp algorithm} \label{alg:hopcroft-karp}
\end{algorithm}

The core of this algorithm is substantially Algorithm \ref{alg:konig}: instead of starting from a maximum matching $M$ to compute the set $T$ (from which follows immediately a minimum vertex cover $C$), we construct the matching and the set $T$ progressively, as follows:

\begin{enumerate}
	\item we construct the directed graph as in Algorithm \ref{alg:directed_graph};
	\item we perform a breadth-first search (following the directed edges) starting from the unmatched vertices in $L$, which terminates when unmatched vertices in $R$ are reached. $l_M$ is the length of these shortest augmenting paths;
	\item the maximal set of vertex-disjoint shortest augmenting paths is computed: we start from an unmatched vertex in $R$ reached in the previous step and perform a depth-first search. Whenever we reach an unmatched vertex in $L$ it means that we found an augmenting path $P$, because we were following the directed edges constructed in the first step. We then add this path to $P$ and resume with the next depth-first search. 
\end{enumerate}

If we use the Hopcroft-Karp algorithm in our sparse graph constructed as in Section \ref{sec:is_graph}, the running time can be even considerably better than the theoretical one: if there are no particularly dense rows and columns, the graph is far from being strongly connected, as each vertex in the graph has just a handful of edges, resulting in fast search phases.

\section{Computation of the priority vector $v$ with the maximum independent set} \label{sec:is_vector}

After having translated our matrix into a graph as in Section \ref{sec:is_graph} and having computed the maximum independent set as described in Section \ref{sec:is_comp}, we still have to compute our priority vector $v$, to be used in the same framework of Section \ref{sec:hot_restart}. Similarly as done for all the methods described in Chapter \ref{chap:methods}, we will distinguish between partition-oblivious heuristics and partition-aware ones.

Let $I \subseteq \{ 0,\dots,m+n-1\}$ be a set of indices. Instead of computing the graph starting from the full matrix $A$, we do it from the submatrix $A(I)$ (i.e. only taking rows and columns in $I$); next, we compute the maximum independent set on the resulting graph using the Hopcroft-Karp algorithm: if we denote by $S_I$ the indices that correspond to this maximum independent set, we always give to this set a high priority, putting it before the remaining indices of $I \setminus S_I$. 

With this in mind, the partition-oblivious version is quite straightforward: we take as $I = \{ 0,\dots,m+n-1\}$, and simply compute

\[
	v := (S_I,I \setminus S_I).
\]

Now, for a partitioned matrix, let $U$ denote the set of uncut indices, and $C$ the set of cut indices. For the partition-aware version of this heuristic we have the following possibilities:

\begin{enumerate}
	\item we compute $S_U$ and have 
\[
	v := (S_U,U \setminus S_U, C);
\]

	\item we compute $S_U$, $S_C$ and have
\[
		v := (S_U, U \setminus S_U, S_C, C \setminus S_C);
\]

	\item we compute $S_U$, then we define $U' := U \setminus S_U$ and compute $S_{C \cup U'}$, having

		\[
			v:= (S_U, S_{C \cup U'}, (C \cup U') \setminus S_{C \cup U'}).
		\]
\end{enumerate}

Note that, by construction, we do not expect these three strategies to be radically different in practice: if at the previous iteration the partitioning was done well, $U$ will be quite big, resulting in similar priority vectors.

In Chapter \ref{chap:experimental_results} we will refer to these three differe heuristic, respectively, as \verb|po_is_1|, \verb|po_is_2| and \verb|po_is_3|.
