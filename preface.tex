This master thesis is, technically, the fruit of my labor over the past nine months, but I would say it is more the result of all the efforts during these two years here at Utrecht University. When I first started, ``scientific computing'' was still a somewhat cloudy concept and when I discovered the field of combinatorial optimization, a new world opened before my eyes: finally my mathematical thinking could be used to solve a number of problems which lie at the very base of computer science, my other great passion. To me, the most amazing aspect of this field is how the basic principles and concepts are surprisingly easy to understand, and yet finding the best solutions is all but trivial. 

I think the simplicity of the problems is well reflected also in my thesis: the goal is efficient sparse matrix partitioning, in order to have a low communication volume during sparse matrix-vector multiplications. Although it might sound complicated, the concept is extremely easy to grasp, even without any familiarity with mathematics: given a very large grid of numbers (matrix), which is mostly empty (sparse), you want to color the nonempty cells either red or blue, such that, at the end, there is roughly the same number of red cells than blue cells; now, the goal is to find a way of doing so such that, in our matrix, rows and columns have, \emph{as much as possible}, only one color. The simplicity of the subject is also well reflected, in my opinion, by the large amount of pictures in this thesis: whenever possible, I tried to come up with simple visualizations that ease the understanding of the discussed concepts.

\section*{Acknowledgments}

First of all, I would like thank my supervisor Rob Bisseling: his guidance was fundamental during these months of research and his suggestions extremely useful. Moreover, I want also to thank him for being ``responsible'' for most of my excitement toward the field. Secondly, I want to thank Bas Fagginger Auer, whose help was vital: every time I was stuck with a problem (mostly with code, but not exclusively) he eagerly, and effectively, helped me with it. I also want to thank him for teaching me a lot, during my first year, in the topic of mathematical writing and this thesis was heavily influenced by him.

In aggiunta, vorrei ringraziare anche la mamma e il papà: è al loro aiuto e supporto che devo davvero tutto. Un grazie anche agli amici, ch\'{e} durante questi mesi mi hanno sempre incoraggiato e hanno dimostrato sempre molta pazienza nei miei confronti. Infine, un grande grazie a Kleopatra, il cui aiuto non poteva essere più prezioso: grazie per avermi sopportato e supportato anche nei momenti più difficili di questi mesi di ricerca, aiutandomi sempre a trovare lo spirito giusto per affrontare i problemi.

Apart from the friends and family back home (I hope the reader will forgive me for having momentarily switched to italian), I also want to thank my friends here in Utrecht, who helped me greatly to adapt to the dutch life and were very useful in helping me keeping stress levels down.

At the end, I would like also to thank you, reader. If you are reading this line after the others, you at least read one page of my thesis, which already makes me proud. Thank you.
